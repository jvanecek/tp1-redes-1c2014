\documentclass[a4paper, 11pt]{article}
\usepackage{amsmath}
\usepackage{amsfonts}
\usepackage{amssymb}
\usepackage{caratula}
\usepackage[spanish, activeacute]{babel}
\usepackage[usenames,dvipsnames]{color}
\usepackage[width=15.5cm, left=3cm, top=2.5cm, height= 24.5cm]{geometry}
\usepackage{graphicx}
\usepackage[utf8]{inputenc}
\usepackage{listings}
\usepackage[all]{xy}
\usepackage{multicol}
\usepackage{subfig}
\usepackage{cancel}
\usepackage{float}
\usepackage{xcolor}
\usepackage{color,hyperref}


%%%%%%%%%%%%%% ALGUNAS MACROS %%%%%%%%%%%%%%
% For \url{SOME_URL}, links SOME_URL to the url SOME_URL
\providecommand*\url[1]{\href{#1}{#1}}

% Same as above, but pretty-prints SOME_URL in teletype fixed-width font
\renewcommand*\url[1]{\href{#1}{\texttt{#1}}}

% Comando para poner el simbolo de Reales
\newcommand{\real}{\hbox{\bf R}}

\providecommand*\code[1]{\texttt{#1}}

%uso: \ponerGrafico{file}{caption}{scale}{label}
\newcommand{\ponerGrafico}[4]
{\begin{figure}[H]
  \centering
  \subfloat{\hspace{-3.5cm}\includegraphics[scale=#3]{#1}}
  \caption{#2} \label{fig:#4}
\end{figure}
}

%\renewcommand{\algorithmiccomment}[1]{\hfill #1}

%%%%%%%%%%%%%%%%%%%%%%%%%%%%%%%%%%%%%%%%%%%%

\materia{Teor\'ia de las Comunicaciones}

\titulo{TP1: Wiretapping}
%\fecha{fecha de entrega}
%\grupo{Nro grupo}
\integrante{Furman, Damián}{936/11}{damian.a.furman@gmail.com}
\integrante{Lambrisca, Santiago}{274/10}{santiagolambrisca@hotmail.com}
\integrante{Marottoli, Daniela}{42/10}{dani.marottoli@gmail.com}
\integrante{Vanecek, Juan}{169-10}{juann.vanecek@hotmail.com}

\include{templates}

\begin{document}
\pagestyle{myheadings}
\maketitle
%\markboth{Nombre materia}{Nombre TP}

\thispagestyle{empty}
\tableofcontents

%\setcounter{section}{-1}
\newpage

\section{Introducción}

Aprovechando las herramientas existentes para el an\'alisis de transferencia de paquetes, como Scapy y Wireshark, nos desarrollamos nuestra propia herramienta que nos permite captar paquetes los paquetes de la redes local a donde estemos conectados. Para poder realizar esto, tuvimos que valernos de una modalidad de uso brindada por la placa de red.

Asi, utilizando la placa de red en modo Promiscuo o Monitor, nos dispusimos a captar los paquetes correspondientes al protocolo ARP (Address Resolution Protocol), con el objetivo de realizar un an\'alisis sobre el intercambio de paquetes de este protocolo realizado en distintas redes, buscando identificar los nodos m\'as significativos e intentando comprender su rol dentro de la red. Este tipo de paquetes es adecuado para este an\'alisis ya que en redes de acceso m\'ultiple son el encargado de traducir direcciones de red (IP) en direcciones de enlace (MAC). Los hosts los env\'ian cuando quieren conocer la ubicaci\'on de cierta IP, y un router est\'a constantemente actualizando su tabla de routeo, por lo que podr\'iamos identificar a estos de acuerdo al flujo de ARPs que corren por la red. 

Vali\'endonos de distintas herramientas de an\'alisis y graficaci\'on hemos realizado este trabajo, obteniendo los resultados y haciendo los an\'alisis presentados a continuaci\'on.

\section{Desarrollo}

En el primer punto nos piden que implementemos una herramienta para escuchar pasivamente una red local. Scapy nos provee una serie de m\'etodos como \texttt{sniff} que ejecuta un callback cada vez que la placa de red recibe un paquete. Luego la clase \texttt{Sniffer} se encarga de parsearlo si es un paquete ARP, y guardarlo convenientemente.  

El paquete esta compuesto, entre otras cosas, por la direcci\'on IP y MAC origen y destino, y el tipo de consulta: \textit{who-has} o \textit{is-at}. 

Como m\'etodo para identificar los routers en la red analizamos tres fuentes de informaci\'on en 5 redes diferentes (2 dom\'esticas, 2 empresariales, 1 p\'ublica). Las fuentes que usamos fueron: 

\begin{enumerate}
 \item {\it IP origen}; evento: IP {\it X} manda un paquete {\it who-has}.
 \item {\it IP destino}; evento: IP {\it X} recibe un paquete {\it who-has}.
 \item {\it IP origen - IP destino}; evento: IP {\it X} manda un paquete who-has a {\it y}.
\end{enumerate}

Para cada una de estas fuentes, la clase \texttt{Sniffer} contiene un diccionario para almacenar cada eventos. 

Una vez que ya tenemos la estructura armada, pusimos a correr el programa unos 30 minutos en cada LAN, un tiempo que consideramos prudente para poder tomar conclusiones. 

Como queremos encontrar los puntos distinguidos en la red, nosotros los vamos a considerar a partir de los eventos que poseamos menos informaci\'on o, lo que es lo mismo, que tengan una mayor probabilidad de que suceda. En particular, nos interesan los eventos $s$ que cumplan $I(s) - H(S) < 0$. 

\section{Gráficos y análisis}

Para las distintas redes utilizadas para llevar a cabo la captura de paquetes, presentamos los datos obtenidos a traves de distintos gr\'aficos, uno para cada fuente de informaci\'on considerada, que nos permiten, no solo mostrar mas claramente los resultados obtenidos, sino que tambi\'en son \'utiles para realizar el analisis de las distintas redes y compararlas entre si.

Consideramos de utilidad, para analizar las fuentes de informaci\'on 1y 2 presentadas en el desarrollo, representar la actividad mostrada por un nodo dentro de cada red graficando la cantidad de informaci\'on que ocurra un evento relacionada a este. Adem\'as decidimos mostrar una linea color rojo que representa la Entropia de la red. El echo de que la informaci\'on brindada por un nodo se encuentre debajo de la linea roja, nos indica que ese nodo tiene una actividad significativa dentro de la red y nos dice que dicho nodo es de importancia a la hora de realizar el analisis.

Por otro lado, para el analisis de la fuente de informaci\'on 3, presentamos un grafo, con nodos de distintos tamaños, donde los nodos representan una IP y los ejes un un paquete que lo referencia como destino o fuente segun la direcci\'on. El tamaño de los nodos representa la cantidad de intercambios en los que participo, viendose asi, los nodos mas significativos graficados con mayor tamaño y siendo facilmente identificables.


\subsection{Red WiFi casa particular 1}
\ponerGrafico{graficos/casa_juan_entropia.png}{Fuente de informaci\'on: IPs que env\'ian}{0.5}{label}
\ponerGrafico{graficos/casa_juan_entropia_rcv.png}{Fuente de informaci\'on: IPs que reciben}{0.5}{label}
Los gr\'aficos presentados anteriormente pertenecen a una casa particular. Podemos observar que uno de los nodos de la red representa una cantidad de unidades de informaci\'on notablemente menor a la de los dem\'as. Siendo que la funci\'on que determina las unidades de informaci\'on para cada evento, en este caso 'paquete enviado por' o 'paquete enviado a', es creciente, mientras mas veces haya ocurrido este evento en el peri\'odo de tiempo que se tom\'o la muestra, menor ser\'a la cantidad de informaci\'on que representa el echo que ese nodo haya enviado o recibido un paquete. Asi, vemos que el nodo cuya informaci\'on se destaca por ser notablemente menor a los demas, es el que m\'as intercambio de paquetes ha realizado. Tentados a pensar que este deberia ser el nodo de la red que representa al Router, pudimos comprobarlo en la configuraci\'on de la red. Otro indicio f\'acil de notar habia sido la direcci\'on IP asociada a este nodo (192.168.0.1)

\ponerGrafico{graficos/casa_juan_grafo.png}{Red WiFi casa dom\'estica 1}{0.6}{label}

En este gr\'afico, podemos identificar claramente a un nodo con gran actividad dentro de la red, coherentemente con los gr\'aficos anteriores, es nuevamente en este gr\'afico el nodo (192.168.0.1) es el de mayor actividad dentro de la red, correspondiendose esta direccion con la del Router.

\subsection{Red WiFi casa particular 2}
\ponerGrafico{graficos/casa_santi_entropia.png}{Fuente de informaci\'on: IPs que env\'ian}{0.5}{label}
\ponerGrafico{graficos/casa_santi_entropia_rcv.png}{Fuente de informaci\'on: IPs que reciben}{0.5}{label}
\ponerGrafico{graficos/casa_santi_grafo.png}{Red WiFi casa dom\'estica 2}{0.5}{label}

\subsection{Red Ethernet Empresa 1 (Recursiva)}
\ponerGrafico{graficos/recursiva_entropia.png}{Fuente de informaci\'on: IPs que env\'ian}{0.5}{label}
\ponerGrafico{graficos/recursiva_entropia_rcv.png}{Fuente de informaci\'on: IPs que reciben}{0.5}{label}
\ponerGrafico{graficos/recursiva_grafo.png}{Red Ethernet de Recursiva}{0.5}{label}

\subsection{Red Ethernet Empresa 2 (ORSNA)}
\ponerGrafico{graficos/orsna_entropia.png}{Fuente de informaci\'on: IPs que env\'ian}{0.5}{label}
\ponerGrafico{graficos/orsna_entropia_rcv.png}{Fuente de informaci\'on: IPs que reciben}{0.5}{label}
\ponerGrafico{graficos/orsna_grafo.png}{Red Ethernet de ORSNA}{0.5}{label}

\subsection{Red WiFi local comercial (McDonalds)}
\ponerGrafico{graficos/mcdonalds_entropia.png}{Fuente de informaci\'on: IPs que env\'ian}{0.5}{label}
\ponerGrafico{graficos/mcdonalds_entropia_rcv.png}{Fuente de informaci\'on: IPs que reciben}{0.5}{label}
\ponerGrafico{graficos/mcdonalds_grafo.png}{Red WiFi de McDonalds}{0.5}{label}
\section{Conclusiones}

\end{document}
