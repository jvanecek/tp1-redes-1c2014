\documentclass[a4paper, 11pt]{article}
\usepackage{amsmath}
\usepackage{amsfonts}
\usepackage{amssymb}
\usepackage{caratula}
\usepackage[spanish, activeacute]{babel}
\usepackage[usenames,dvipsnames]{color}
\usepackage[width=15.5cm, left=3cm, top=2.5cm, height= 24.5cm]{geometry}
\usepackage{graphicx}
\usepackage[utf8]{inputenc}
\usepackage{listings}
\usepackage[all]{xy}
\usepackage{multicol}
\usepackage{subfig}
\usepackage{cancel}
\usepackage{float}
\usepackage{xcolor}
\usepackage{color,hyperref}


%%%%%%%%%%%%%% ALGUNAS MACROS %%%%%%%%%%%%%%
% For \url{SOME_URL}, links SOME_URL to the url SOME_URL
\providecommand*\url[1]{\href{#1}{#1}}

% Same as above, but pretty-prints SOME_URL in teletype fixed-width font
\renewcommand*\url[1]{\href{#1}{\texttt{#1}}}

% Comando para poner el simbolo de Reales
\newcommand{\real}{\hbox{\bf R}}

\providecommand*\code[1]{\texttt{#1}}

%uso: \ponerGrafico{file}{caption}{scale}{label}
\newcommand{\ponerGrafico}[4]
{\begin{figure}[H]
  \centering
  \subfloat{\hspace{-3.5cm}\includegraphics[scale=#3]{#1}}
  \caption{#2} \label{fig:#4}
\end{figure}
}

%\renewcommand{\algorithmiccomment}[1]{\hfill #1}

%%%%%%%%%%%%%%%%%%%%%%%%%%%%%%%%%%%%%%%%%%%%

\materia{Teor\'ia de las Comunicaciones}

\titulo{TP1: Wiretapping}
%\fecha{fecha de entrega}
%\grupo{Nro grupo}
\integrante{Furman, Damián}{936/11}{damian.a.furman@gmail.com}
\integrante{Lambrisca, Santiago}{274/10}{santiagolambrisca@hotmail.com}
\integrante{Marottoli, Daniela}{42/10}{dani.marottoli@gmail.com}
\integrante{Vanecek, Juan}{169-10}{juann.vanecek@hotmail.com}

\include{templates}

\begin{document}
\pagestyle{myheadings}
\maketitle
%\markboth{Nombre materia}{Nombre TP}

\thispagestyle{empty}
\tableofcontents

%\setcounter{section}{-1}
\newpage

\section{Introducción}

Aprovechando las herramientas existentes para el an\'alisis de transferencia de paquetes, como Scapy y Wireshark, nos desarrollamos nuestra propia herramienta que nos permite captar paquetes de distintas redes inal\'ambricas a\'un cuando estos paquetes no estaban destinados a nuestro host. Para poder realizar esto, tuvimos que valernos de una modalidad de uso brindada por la placa de red.
Asi, utilizando la placa de red en modo Promiscuo o Monitor, nos dispusimos a captar los paquetes correspondientes al protocolo ARP (Address Resolution Protocol), con el objetivo de realizar un an\'alisis sobre el intercambio de paquetes de este protocolo realizado en distintas redes, buscando identificar los nodos m\'as significativos e intentando comprender su rol dentro de la red.
Valiendonos de distintas herramientas de an\'alisis y graficaci\'on hemos realizado este trabajo, obteniendo los resultados y haciendo los an\'alisis presentados a continuaci\'on.

\section{Desarrollo}

En el primer punto nos piden que implementemos una herramienta para escuchar pasivamente una red local. Y para ello nos basamos en 

\section{Gráficos y análisis}

\subsection{Red WiFi casa particular 1}
\ponerGrafico{graficos/casa_juan_entropia.png}{Fuente de informaci\'on: IPs que env\'ian}{0.5}{label}
\ponerGrafico{graficos/casa_juan_entropia_rcv.png}{Fuente de informaci\'on: IPs que reciben}{0.5}{label}
\ponerGrafico{graficos/casa_juan_grafo.png}{Red WiFi casa dom\'estica 1}{0.6}{label}

\subsection{Red WiFi casa particular 2}
\ponerGrafico{graficos/casa_santi_entropia.png}{Fuente de informaci\'on: IPs que env\'ian}{0.5}{label}
\ponerGrafico{graficos/casa_santi_entropia_rcv.png}{Fuente de informaci\'on: IPs que reciben}{0.5}{label}
\ponerGrafico{graficos/casa_santi_grafo.png}{Red WiFi casa dom\'estica 2}{0.5}{label}

\subsection{Red Ethernet Empresa 1}
\ponerGrafico{graficos/recursiva_entropia.png}{Fuente de informaci\'on: IPs que env\'ian}{0.5}{label}
\ponerGrafico{graficos/recursiva_entropia_rcv.png}{Fuente de informaci\'on: IPs que reciben}{0.5}{label}
\ponerGrafico{graficos/recursiva_grafo.png}{Red Ethernet de Recursiva}{0.5}{label}

\subsection{Red Ethernet Empresa 2}
\ponerGrafico{graficos/orsna_entropia.png}{Fuente de informaci\'on: IPs que env\'ian}{0.5}{label}
\ponerGrafico{graficos/orsna_entropia_rcv.png}{Fuente de informaci\'on: IPs que reciben}{0.5}{label}
\ponerGrafico{graficos/orsna_grafo.png}{Red Ethernet de ORSNA}{0.5}{label}

\subsection{Red WiFi local comercial}
\ponerGrafico{graficos/mcdonalds_entropia.png}{Fuente de informaci\'on: IPs que env\'ian}{0.5}{label}
\ponerGrafico{graficos/mcdonalds_entropia_rcv.png}{Fuente de informaci\'on: IPs que reciben}{0.5}{label}
\ponerGrafico{graficos/mcdonalds_grafo.png}{Red WiFi de McDonalds}{0.5}{label}
\section{Conclusiones}

\end{document}
