\documentclass[a4paper, 11pt]{article}
\usepackage{amsmath}
\usepackage{amsfonts}
\usepackage{amssymb}
\usepackage{caratula}
\usepackage[spanish, activeacute]{babel}
\usepackage[usenames,dvipsnames]{color}
\usepackage[width=15.5cm, left=3cm, top=2.5cm, height= 24.5cm]{geometry}
\usepackage{graphicx}
\usepackage[utf8]{inputenc}
\usepackage{listings}
\usepackage[all]{xy}
\usepackage{multicol}
\usepackage{subfig}
\usepackage{cancel}
\usepackage{float}
\usepackage{xcolor}
\usepackage{color,hyperref}


%%%%%%%%%%%%%% ALGUNAS MACROS %%%%%%%%%%%%%%
% For \url{SOME_URL}, links SOME_URL to the url SOME_URL
\providecommand*\url[1]{\href{#1}{#1}}

% Same as above, but pretty-prints SOME_URL in teletype fixed-width font
\renewcommand*\url[1]{\href{#1}{\texttt{#1}}}

% Comando para poner el simbolo de Reales
\newcommand{\real}{\hbox{\bf R}}

\providecommand*\code[1]{\texttt{#1}}

%uso: \ponerGrafico{file}{caption}{scale}{label}
\newcommand{\ponerGrafico}[4]
{\begin{figure}[H]
  \centering
  \subfloat{\hspace{-3.5cm}\includegraphics[scale=#3]{#1}}
  \caption{#2} \label{fig:#4}
\end{figure}
}

%\renewcommand{\algorithmiccomment}[1]{\hfill #1}

%%%%%%%%%%%%%%%%%%%%%%%%%%%%%%%%%%%%%%%%%%%%

\materia{Teor\'ia de las Comunicaciones}

\titulo{TP1: Wiretapping}
%\fecha{fecha de entrega}
%\grupo{Nro grupo}
\integrante{Furman, Damián}{936/11}{damian.a.furman@gmail.com}
\integrante{Lambrisca, Santiago}{274/10}{santiagolambrisca@hotmail.com}
\integrante{Marottoli, Daniela}{42/10}{dani.marottoli@gmail.com}
\integrante{Vanecek, Juan}{169-10}{juann.vanecek@hotmail.com}

\include{templates}

\begin{document}
\pagestyle{myheadings}
\maketitle
%\markboth{Nombre materia}{Nombre TP}

\thispagestyle{empty}
\tableofcontents

%\setcounter{section}{-1}
\newpage

\section{Introducción}

Aprovechando las herramientas existentes para el an\'alisis de transferencia de paquetes, como Scapy y Wireshark, nos desarrollamos nuestra propia herramienta que nos permite captar los paquetes de la redes local a donde estemos conectados. Para poder realizar esto, tuvimos que valernos de una modalidad de uso brindada por la placa de red.

As\'i, utilizando la placa de red en modo Promiscuo o Monitor, nos dispusimos a captar los paquetes correspondientes al protocolo ARP (Address Resolution Protocol), con el objetivo de realizar un an\'alisis sobre el intercambio de paquetes de este protocolo realizado en distintas redes, buscando identificar los nodos m\'as significativos e intentando comprender su rol dentro de la red. Este tipo de paquetes es adecuado para este an\'alisis ya que en redes de acceso m\'ultiple es el encargado de traducir direcciones de red (IP) en direcciones de enlace (MAC). Los hosts los env\'ian cuando quieren conocer la ubicaci\'on de cierta IP, y un router est\'a constantemente actualizando su tabla de routeo, por lo que podr\'iamos identificar a \'estos de acuerdo al flujo de ARPs que corren por la red. 

Vali\'endonos de distintas herramientas de an\'alisis y graficaci\'on hemos realizado este trabajo, obteniendo los resultados presentados a continuaci\'on.

\section{Desarrollo}

En el primer punto se pide implementar una herramienta para escuchar pasivamente una red local. Scapy nos provee una serie de m\'etodos como \texttt{sniff} que ejecuta un callback cada vez que la placa de red recibe un paquete. Luego la clase \texttt{Sniffer} se encarga de parsearlo si es un paquete ARP, y guardarlo convenientemente.  

El paquete est\'a compuesto, entre otras cosas, por la direcci\'on IP y MAC origen y destino, y el tipo de consulta: \textit{who-has} o \textit{is-at}. 

Como m\'etodo para identificar los routers en la red analizamos tres fuentes de informaci\'on en 5 redes diferentes (dos dom\'esticas, una empresarial, una de un organismo, y una p\'ublica). Las fuentes que usamos fueron: 

\begin{enumerate}
 \item {\it IP origen}; evento: IP {\it X} origen de un paquete {\it who-has}.
 \item {\it IP destino}; evento: IP {\it X} destino de un paquete {\it who-has}.
 \item {\it IP origen - IP destino}; evento: IP {\it X} manda un paquete who-has a {\it y}.
\end{enumerate}

Para cada una de estas fuentes, la clase \texttt{Sniffer} contiene un diccionario para almacenar cada evento. 

Una vez que ya tenemos la estructura armada, pusimos a correr el programa unos 30 minutos en cada LAN, un tiempo que consideramos prudente para poder tomar conclusiones. 

Como queremos encontrar los puntos distinguidos en la red, los vamos a considerar a partir de los eventos que posean menos informaci\'on o, lo que es lo mismo, que tengan una mayor probabilidad de que suceda. En particular, nos interesan los eventos $s$ que cumplan $I(s) - H(S) < 0$. 

\section{Gráficos y análisis}

Para las distintas redes utilizadas, presentamos los datos obtenidos a trav\'es de distintos gr\'aficos, uno para cada fuente de informaci\'on considerada, que nos permiten, no solo mostrar mas claramente los resultados obtenidos, sino que tambi\'en son \'utiles para realizar el an\'alisis de las distintas redes y compararlas entre s\'i.

Consideramos de utilidad, para analizar las fuentes de informaci\'on 1 y 2 presentadas en el desarrollo, representar la actividad mostrada por cada nodo dentro de una red graficando la cantidad de informaci\'on  proporcionada por el evento relacionado a este. Adem\'as decidimos mostrar una l\'inea color rojo que representa la Entrop\'ia de la red. El echo de que la informaci\'on brindada por un nodo se encuentre debajo de la linea roja, nos indica que ese nodo tiene una actividad significativa dentro de la red y nos dice que dicho nodo es de importancia a la hora de realizar el an\'alisis.

Por otro lado, para el analisis de la fuente de informaci\'on 3, presentamos un grafo, con nodos de distintos tamaños, donde los nodos representan una IP y los ejes un un paquete que lo referencia como destino o fuente seg\'un la direcci\'on. El tamaño de los nodos representa la cantidad de intercambios en los que particip\'o, viendose as\'i, los nodos mas significativos graficados con mayor tamaño y siendo facilmente identificables.


\subsection{Red WiFi casa particular 1}

\ponerGrafico{graficos/casa_juan_entropia.png}{Fuente de informaci\'on: IPs que env\'ian}{0.5}{label}
\ponerGrafico{graficos/casa_juan_entropia_rcv.png}{Fuente de informaci\'on: IPs que reciben}{0.5}{label}

Los gr\'aficos presentados anteriormente pertenecen a una red dom\'estica. Podemos observar que uno de los nodos de la red representa una cantidad de unidades de informaci\'on notablemente menor a la de los dem\'as.
Siendo que la funci\'on que determina las unidades de informaci\'on para cada evento, es decreciente, mientras m\'as veces haya ocurrido este evento en el peri\'odo de tiempo que se tom\'o la muestra, menor ser\'a la cantidad de informaci\'on que representa el hecho que ese evento ocurra. As\'i, vemos que el evento cuya informaci\'on se destaca por ser notablemente menor a los dem\'as, es el que m\'as intercambio de paquetes ha realizado. Tentados a pensar que \'este deber\'ia ser el nodo de la red que representa al Router, pudimos comprobarlo en la configuraci\'on de la red. Otro indicio f\'acil de notar hab\'ia sido la direcci\'on IP asociada a este nodo (192.168.0.1)

\ponerGrafico{graficos/casa_juan_grafo.png}{Red WiFi casa dom\'estica 1}{0.6}{label}

En este gr\'afico, podemos identificar claramente a un nodo con gran actividad dentro de la red, coherentemente con los gr\'aficos anteriores, es nuevamente en este gr\'afico el nodo (192.168.0.1) es el de mayor actividad dentro de la red, correspondi\'endose esta direcci\'on con la del Router.

\subsection{Red WiFi casa particular 2}
\ponerGrafico{graficos/casa_santi_entropia.png}{Fuente de informaci\'on: IPs que env\'ian}{0.5}{label}
\ponerGrafico{graficos/casa_santi_entropia_rcv.png}{Fuente de informaci\'on: IPs que reciben}{0.5}{label}

Estos gr\'aficos se corresponden con otra red de una casa. Sin embargo, nos devuelve algunos resultados que no est\'abamos esperando.
En el primer gr\'afico nos encontramos con un evento distinguido, el cual se corresponde con la IP  192.168.0.27, sin embargo, si miramos el segundo gr\'afico el evento distinguido se corresponde con otra direcci\'on IP, la 192.168.0.1. Esto nos plantea un interrogante sobre cu\'al se corresponde al Router. Nuestros conocimientos previos nos llevan a pensar que la segunda direcci\'on se corresponde con el Router, y nos encargamos de chequearlo. La duda paso a ser, a qu\'e host pertenec\'ia la otra IP. Nos encontramos con que esta IP se corresponde al Servidor Web, Apache, utilizado localmente.

\ponerGrafico{graficos/casa_santi_grafo.png}{Red WiFi casa dom\'estica 2}{0.5}{label}

Nuevamente, y coherentemente con lo mostrado en los gr\'aficos anteriores, nos encontramos con un nodo significativo, pero que no se corresponde con el Router, sino con el Servidor Web. Sin embargo, debido a la cantidad de aristas con las que se conecta uno de los nodos, nos permite pensar que ese es el Router y lo comprobamos ya que su IP es 192.168.0.1

\subsection{Red Ethernet Empresa (Recursiva)}
\ponerGrafico{graficos/recursiva_entropia.png}{Fuente de informaci\'on: IPs que env\'ian}{0.5}{label}
\ponerGrafico{graficos/recursiva_entropia_rcv.png}{Fuente de informaci\'on: IPs que reciben}{0.5}{label}
\ponerGrafico{graficos/recursiva_grafo.png}{Red Ethernet de Recursiva}{0.5}{label}

\subsection{Red Ethernet Organismo (ORSNA)}
\ponerGrafico{graficos/orsna_entropia.png}{Fuente de informaci\'on: IPs que env\'ian}{0.5}{label}
\ponerGrafico{graficos/orsna_entropia_rcv.png}{Fuente de informaci\'on: IPs que reciben}{0.5}{orsna_entr_rcv}

A diferencia de las LAN anteriores, podemos notar que no est\'a tan expl\'icita la IP del router. Con ambas fuentes de informaci\'on encontramos distintos hosts distinguibles. Rastreando a d\'onde pertenece cada dirección, encontramos que la IP 10.0.21.19 pertenec\'ia a una impresora que dej\'o de funcionar y fue sacada del organismo, pero no fue sacada de la configuraci\'on de dispositivos de algunas computadoras, que son aquellas que realizan llamados who-has constantemente a esa direcci\'on buscando la impresora perdida. Es por ello que esta direcci\'on est\'a por debajo de la entrop\'ia en el gr\'afico \ref{fig:orsna_entr_rcv}, debido a la cantidad de informaci\'on que lleva este evento. 

Adem\'as, tampoco se pudo rastrear a qu\'e hab\'ia pertenecido la direcci\'on 10.0.21.20, otra direcci\'on distinguible en los destinos, dado que actualmente no est\'a asignada a ning\'un hardware. Esta direcci\'on fue requerida constantemente por la IP 10.0.20.57 (al rededor de mil veces en 30 minutos). Esto es lo que hace que sea una fuente distinguida, pero no podemos deducir por qu\'e, dado que nos falta informaci\'on.

Con todos los datos sucios debido a los inconvenientes mencionados anteriormente, no se puede deducir f\'acilmente  cu\'al es el router de la red, aunque rastre\'andolo pudimos comprobar que este se encuentra en la direcci\'on 10.0.21.1.

\ponerGrafico{graficos/orsna_grafo.png}{Red Ethernet de ORSNA}{0.5}{label}

En este gr\'afico podemos ver nuevamente lo que mencion\'abamos antes. Las direcci\'on IP 10.0.21.57 es la que m\'as paquetes who-has mand\'o a la IP fantasma (10.0.21.20), y luego todas se mantienen m\'as o menos igual debido que todas mandaron paquetes a la IP 10.0.21.19 que pertenec\'ia a la impresora.

\subsection{Red WiFi local comercial (McDonalds)}
\ponerGrafico{graficos/mcdonalds_entropia.png}{Fuente de informaci\'on: IPs que env\'ian}{0.5}{mc_entropia}
\ponerGrafico{graficos/mcdonalds_entropia_rcv.png}{Fuente de informaci\'on: IPs que reciben}{0.5}{mc_entropia_rcv}

Por \'ultimo, pusimos a correr nuestro programa en la red WiFi del local comercial McDondalds y al analizarlo se ve claramente un solo nodo distinguido en el gr\'afico \ref{fig:mc_entropia} al que podemos asumir sin dudar como el router. En el lapso de 40 minutos, el host 172.17.58.1 mand\'o 1372 ARP requests en contraste con los otros hosts que mandaron como mucho 38.  

La otra fuente no aporta muchos datos ya que todos los nodos reciben muchos paquetes por igual. 

\ponerGrafico{graficos/mcdonalds_grafo.png}{Red WiFi de McDonalds}{0.5}{label}

\section{Conclusiones}

Cuando presentamos el trabajo pr\'actico, marcabamos la influencia que tiene el rol de cada dispositivo con respecto al uso del protocolo ARP. As\'i, esperabamos que el Router sea un nodo distinguido en las redes que ibamos a analizar.
En los distintos casos, nos encontramos con diferentes resultados. Algunos de ellos se correspond\'ian con lo que esper\'abamos, pero en otros nos encontramos con algunos resultados inesperados.

Pudimos ver, que si bien el Router participa en una cantidad importante de los paquetes ARP que circulan en la red, podemos encontrarnos con distintos Hosts que cobran importancia. Sin embargo, dejando de lado el an\'alisis meramente num\'erico, se puede observar que los nodos que representan a los Routers, en la gran mayoria de los casos, se encuentran rodeados por varios nodos, y presentan muchas aristas, con respecto a los dem\'as.

Sin lugar a dudas, la gran variedad de redes, las distintas formas de configurarlas y la diversidad de Hosts que podemos encontrarnos, nos muestran que no existe una regla que nos diga, que rol cumple cada nodo. En las casas particulares y lugares p\'ublicos es efectivo usar la fuente de informaci\'on que usa el evento de ip origen, porque hay un solo gateway y los dem\'as hosts son en general terminales de usuarios normales como pc, notebooks o celulares que no interfieren en el an\'alisis. A diferencia de una red en un lugar de trabajo, donde podemos encontrar distintos routers, y terminales de administradores que pueden estar haciendo mantenimiento o su propio estudio de la red, generando tr\'afico ARP para encontrar hosts ca\'idos, por ejemplo, y que influye en los resultados finales, como ya vimos. 

A pesar de esto, lo que si podemos encontrar es una generalidad, y ver que nuestra hip\'otesis primigenia de que el Router sea quien m\'as utiliza el protocolo ARP es acertada, y en la mayor\'ia de los escenarios es efectivo.

\end{document}
